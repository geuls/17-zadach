\documentclass[10pt, a3paper]{scrartcl}
\usepackage[utf8]{inputenc}
\usepackage[english,russian]{babel}
\usepackage{indentfirst}
\usepackage{misccorr}
\usepackage{graphicx}
\usepackage{amsmath}
\linespread{2.3}
\pagestyle{empty}
\usepackage{geometry} \geometry{verbose,a3paper,tmargin=2cm,bmargin=2cm,lmargin=2.5cm,rmargin=1.5cm}
\begin{document}
\huge \textbf{17 уравнений, которые изменили мир(по мнению Иэна Стюарта).
		}
	\\
	\\ 
	\\
	\date{}
	\large
\begin{tabular}{p{8cm}  p{7cm} p{10cm}}   
   \textbf{1)   Теорема Пифагора}  &
 	  
 	   {$a^2+b^2=c^2$} & {Пифагор, 530 лет до нашей эры} \\
 
      \textbf{2)   Логарифмы} &
    
     {$\log xy = \log x + \log y$} & {Джон Непер, 1610} \\      
  
      \textbf{3)   Дифференциальное исчисление} &
       
   {$\frac{df}{dt} = \lim \limits_{h\to 0} \frac{f(t+h)-f(t)}{h}$} &  {Ньютон, 1668}\\
   
      \textbf{4)   Закон гравитации} &
     
     {$F = G\dfrac{m_{1} m_{2}}{r^2}$} & {Ньютон, 1687}\\
   
      \textbf{5)   Квадратный корень из
      	минус единицы} &
      {$i^{2} = -1$} &  {Эйлер, 1750}\\
    
      \textbf{6)   Формула Эйлера для
      	многогранников} &
      {$\mathrm{V}-\mathrm{E}+\mathrm{F}=2$} &  {Эйлер, 1751}\\
  
      \textbf{7)   Нормальное распределение} &
      {$\phi \left(x\right)=\frac{1}{\sqrt{2\pi p}}{C}^{\frac{{\left(x-p\right)}^{2}}{2{p}^{2}}}$} & {К.Ф. Гаусс, 1810} \\
  
      \textbf{8)   Волновое уравнение} &
      {$\frac{{\partial }^{2}u}{\partial {t}^{2}}={C}^{2}\frac{{\partial }^{2}U}{\partial {x}^{2}}$} & {Ж.Л. Д'Аламбер, 1746}\\     
   
      \textbf{9)   Преобразование Фурье} &
      {$f(a)=\underset{\infty }{\overset{\infty }{\int }}f\left(x\right){e}^{-2\pi xa}dx$}  &  {Ж. Фурье, 1822}\\
   
      \textbf{10)   Уравнение
      	Навье-Стокса} &
      {$p(\dfrac{\partial v}{\partial t}+ v \cdot \nabla v) = - \nabla p + \nabla \cdot T + f$}   & {А. Навье, Д. Стокс, 1845}\\
  
      \textbf{11)   Уравнения Максвелла} &
      {
      	$\nabla \cdot E = \dfrac{p}{\epsilon_0} $  
      	
      	$ \nabla \cdot H = 0$
      	
      	$\nabla  \times E = - \dfrac{1}{e} \dfrac{\partial H}{\partial t} $
      	
      	$\nabla  \times H = \dfrac{1}{e} \dfrac{\partial E}{\partial t} $
      } &  {Д.К. Максвелл, 1865}\\
   
      \textbf{12)   Второй закон
      	термодинамики} &
      {$dS\ge 0$} &  {Л. Больцман, 1874}\\
  
      \textbf{13)   Относительность} &
      {$E=mc^2$} & {Эйнштейн, 1905} \\
  
      \textbf{14)   Уравнение
      	Шредингера} &
      {$ih \dfrac{\partial}{\partial t} \psi = H \psi$} & {Э. Шрёдингер, 1927} \\
   
      \textbf{15)   Теория информации} &
      {$H = \sum p(x) \log p(x)$} & {К.Э. Шеннон, 1949}\\
   
      \textbf{16)   Теория хаоса} &
      {$x_{t+1}=kx_{t}(1-x_{t})$}  & {Р.Мей, 1975} \\

      \textbf{17)   Уравнение
      	Блэка — Шоулза} &
      {$\dfrac{1}{2} \sigma^2 S^2 \dfrac{\partial^2 V}{\partial S^2} + rS \dfrac{\partial V}{\partial S} + \dfrac{\partial V}{\partial t} - rV = 0$} & {Ф. Блек, М. Шоулз, 1990}\\
\end{tabular} 
\end{document}